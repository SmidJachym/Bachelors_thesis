\documentclass[12pt,a4paper, fleqn]{report}
\usepackage[utf8]{inputenc}
\usepackage[czech]{babel}
\usepackage[T1]{fontenc}
\usepackage{amsmath}
\usepackage{amsfonts}
\usepackage{amssymb}
\usepackage{graphicx}
\usepackage{natbib}
\usepackage[left=3.5cm,right=2cm,top=2.5cm,bottom=2.5cm]{geometry}
\usepackage{afterpage}
\date{}
\title{\textbf{Bifurkace v obyčejných diferenciálních rovnicích}\\[2ex]\large{Ústav technické matematiky\\ FS ČVUT v Praze}}
\author{Jáchym Šmíd}


\begin{document}

%---------------------------- styling-------------------------

\setlength{\mathindent}{15pt}
\pagestyle{headings}
\renewcommand{\familydefault}{latinmodernroman}
\newcommand\myemptypage{
    \null
    \thispagestyle{empty}
    \newpage
    }
\setlength{\parskip}{0.5em}
    
%------------------------------------------------------------

\maketitle
\thispagestyle{empty}
\pagebreak

\myemptypage

\thispagestyle{empty}
\section*{Anotace}
moje super anotace
\section*{Anotation}
my awesome anotation
\section*{Klíčová slova}
sjdls, sdjvlsd, lksdlv, lksda
\section*{Keywords}
sův, kslr, jada.jda
\section*{Poděkování}
Moc děkuji Váam
\pagebreak

\thispagestyle{empty}
\tableofcontents
\pagebreak

\thispagestyle{empty}
\section*{Přehled použitých značek}
\begin{align*}
q_A & - \text{tepelný tok } & [q_A] = W\cdot m^{-2}\\
k & - \text{součinitel kondukce} & [k] = W\cdot m^{-1}\cdot K^{-1}\\
\nabla & - \text{gradient, definován jako}\ \nabla = \displaystyle\left(\frac{\partial}{\partial x};\frac{\partial}{\partial y};\frac{\partial}{\partial x}\right) & [\nabla] = m^{-1}\\
T & - \text{aboslutní teplota} & [T] = K\\
v_{\infty} & - \text{rychlost náběhu} & [v_{\infty}] = m\cdot s^{-1}\\
c & - \text{rychlost} & [c] = m\cdot s^{-1}\\
u & - \text{měrná vnitřní energie} & [u] = J\cdot kg^{-1} \cdot K^{-1}\\
h & - \text{měrná entalpie} & [h] = J\cdot kg^{-1} \cdot K^{-1}\\
p & - \text{tlak} & [p] = N\cdot m^{-2}\\
g & - \text{gravitační zrychlení},\ g \approx 9.81 & [g] = m \cdot s^{-2}\\
P & - \text{výkon} & [P] = W\\ 
W & - \text{práce} & [W] = J\\
Q & - \text{teplo} & [Q] = J\\
q & - \text{měrné teplo} & [q] = J \cdot kg^{-1}\\
\end{align*}

\pagebreak


\chapter{Úvod}
\section{jada}
Miscellaneous information
About the LaTeX Font Catalogue
Font documentation
Packages that provide math fonts
\subsection{parts}
About the various version of the Computer Modern fonts
Last update of something in the catalogue: 2021-05-03
Meet The LaTeX Font Catalogue on Facebook:  http://www.facebook.com/LaTeXFontCatalogue


\[
\iint_{\Omega}u(x,y,t)d\Omega = C
\]


\end{document}